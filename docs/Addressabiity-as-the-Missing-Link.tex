\documentclass[11pt]{article}

% Packages
\usepackage[utf8]{inputenc}
\usepackage{hyperref}
\usepackage{graphicx}
\usepackage{amsmath}
\usepackage{listings}
\usepackage{minted}
\usepackage{tikz}
\usepackage{xcolor}

% Document metadata
\title{Addressability is The Missing Link (in the Web of Trust)\\[0.5em]\Large LinkedClaims: An Open Standard for Addressable Claims}
\author{
    Golda Velez\thanks{Linked Trust} \and
    Second Author\thanks{Organization 2} \and
    Additional Authors\thanks{Their Organizations}
}
\date{\today}

\begin{document}

\maketitle

\begin{abstract}
Trust and credibility of information is critical to a functioning society. While cryptographic signing and blockchain validation provide technical verification, they are insufficient to predict the truth of an assertion. This paper introduces LinkedClaims, an open standard for addressable claims that enables cross-domain credibility assessment. We present the technical specification, analyze existing implementations across diverse domains, and outline the ecosystem development pathway.
\end{abstract}

\section{Executive Summary}

\subsection{Cross-system Trust Verification Challenge}
\subsection{Open Standard Solution}
\subsection{Technical Requirements Overview}

\section{Technical Architecture \& Security}
\subsection{URI-addressability Requirements}
\subsection{Hashability and Cryptographic Standards}
\subsection{Security Analysis}
\subsection{Integration Flexibility}
\subsection{Privacy Preservation Architecture}

\section{Implementation Analysis}
\subsection{Compliance Testing Framework}
\subsection{Implementation Examples}
\subsubsection{Skill Credentials System}
\begin{itemize}
    \item US Chamber of Commerce Implementation
    \item Cross-border Recognition Patterns
    \item Integration with Educational Systems
\end{itemize}

\subsubsection{Supply Chain Verification}
\begin{itemize}
    \item UN CRM Implementation
    \item Traceability Architecture
    \item Multi-party Validation Flows
\end{itemize}

\subsubsection{Environmental Monitoring}
\begin{itemize}
    \item Open Forest Protocol Integration
    \item Sensor Data Validation
    \item Impact Assessment Framework
\end{itemize}

\subsubsection{Philanthropic Impact Tracking}
\begin{itemize}
    \item Grant Outcome Verification
    \item Impact Measurement Standards
    \item Cross-organization Validation
\end{itemize}

\subsubsection{Blockchain Wallet Credentials}
\begin{itemize}
    \item Identity Verification Patterns
    \item Privacy-preserving Claims
    \item Cross-chain Compatibility
\end{itemize}

\subsection{Integration Patterns}
\subsubsection{Blockchain Compatibility}
\subsubsection{Verifiable Credentials Alignment}
\subsubsection{Web Embedding Methods}

\subsection{Open Source Libraries}
\subsection{Private Data Handling Patterns}

\section{Community Applications \& Use Cases}
\subsection{Current Implementations}
\subsection{Network Effects}
\subsection{Cross-domain Benefits}
\subsection{Public/Private Data Models}

\section{Ecosystem Development}
\subsection{Integration Pathways}
\subsection{Partnership Examples}
\subsection{Network Growth Patterns}
\subsection{Commercial Implementation Opportunities}
\subsection{Value Capture Models}
\subsection{Marketplace Dynamics}

\section{Technical Roadmap}
\subsection{Standards Development}
\subsection{Research Directions}
\subsection{Open Challenges}
\subsection{Community Contribution Framework}
\subsection{Commercial Extension Patterns}

\bibliographystyle{plain}
\bibliography{references}

\appendix
\section{Technical Implementation Details}
\section{Integration Examples}
\section{Security Considerations}

\end{document}
